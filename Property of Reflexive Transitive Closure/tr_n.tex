\documentclass[a4paper,12pt]{article}
\usepackage[utf8]{inputenc}
\usepackage{amsthm}
\usepackage{amsmath}
\usepackage{xcolor}
\newtheorem{definition}{Definition}
\newtheorem{claim}{Claim}
\newtheorem{theorem}{Theorem}
\usepackage{setspace}
\usepackage[hyphens]{url}
\newtheorem{example}{Example}
\newtheorem{condition}{Condition}
\newtheorem{proposition}{Proposition}
\usepackage{graphicx}

\definecolor{mygreen}{RGB}{0,150,0}
\newtheorem{lemma}{Lemma}
\usepackage{color}
\usepackage{calrsfs}
\usepackage{latexsym}
\usepackage{url}
\usepackage{listings}
\usepackage{mathrsfs}
\usepackage[colorlinks=true]{hyperref}
\hypersetup{
    colorlinks,
    citecolor=black,
    filecolor=black,
    linkcolor=blue,
    urlcolor=blue
}
\usepackage{ulem}
\def\st{\noindent}
\renewcommand\em{\it}
\renewcommand\emph{\textit}
\newcommand\red[1]{{\color{red}#1}}
\newcommand\blue[1]{{\color{blue}#1}}
\newcommand\green[1]{{\color{mygreen}#1}}
\newcommand\cancelr[1]{{\color{red}\sout{#1}}}
\newcommand\cancelg[1]{{\color{mygreen}\sout{#1}}}
%\newcommmand{\red}[1]{\textcolor{red}{#1}}
%\renewcommand\red[1]{{\color{red}#1}}
%opening
\title{}
\author{}
\def\no{{ not}\;}
\begin{document}
\begin{definition}
{\rm
Let $R$ be a binary relation on a set $S$. We say that $R^+$ is the reflexive transitive closure of $R$
 if the following conditions are satisfied: 
\begin{enumerate}
\item $R \subseteq R^+$ and 
\item for every transitive reflexive relation $R^*$ such that $R \subseteq R^*$, we have $R^+ \subseteq R^*$ and
\item \red{$R^+$ is reflexive and transitive.} 
\end{enumerate}
}
\hfill $\Box$   
\end{definition}

\begin{theorem}
{\rm
Let $R$ be a binary relation on a a set $S$.
Let $R^@$ be another binary relation on $S$ such that for every $A,B \in S$ we have $R^@(A,B)$ iff there exists a sequence $C_1,\ldots,C_n$ of elements of $S$ such that 
\begin{enumerate}
\item $C_1 = A$
\item $C_n = B$
\item for every $i \in \{1..n-1\}$, $R(C_i, C_{i+1})$
\end{enumerate} 
We have:
$$ R^@ = R^+$$
\hfill $\Box$
}
\end{theorem}
\begin{proof}
We first show
\begin{equation}\label{eqr1}
R\subseteq R^@
\end{equation}
Indeed, suppose $(A,B) \in R$. Then we have a two-element sequence $C_1 = A, C_2 = B$ such that $(C_1,C_2) \in R$. Thus, by definition of $R^@$, $(A,B) \in R^@$.

\noindent
Next, we show

\begin{equation}\label{eqr2}
R^@ \mbox{ is reflexive}
\end{equation}

\noindent
Indeed, consider an element  $A \in S$. Then we have a sequence consisting of one element, $C_1 = A$. It is easy to see that conditions 1-3 of the definition of $R^@$ are satisfied for the pair $(A,A)$.
Therefore, $R^@(A,A)$ holds for every $A \in S$ and we have (\ref{eqr2}). 

\noindent
Next we show 
\begin{equation}\label{eqr3}
R^@ \mbox{ is transitive}
\end{equation}

\noindent
Let $A,B,C$ be three elements of $S$ (not necessarily distinct) such that 

\begin{equation}\label{eqa1}
R^@(A,B)
\end{equation}
and
\begin{equation}\label{eqa2}
R^@(B,C)
\end{equation}
To prove (\ref{eqr2}), it is sufficent to show

\begin{equation}\label{eqa3}
R^@(A,C)
\end{equation}
From (\ref{eqa1}), by definition of $R^@$, there exists a sequence $C^1_1,\ldots,C_n^1$ of elements of $S$ such that
\begin{equation}\label{boo}
C^1_1 = A
\end{equation}
\begin{equation}
C^1_n = B
\end{equation}
\begin{equation}
\mbox{for every $i \in \{1..n-1\}$,} R(C^1_i,C^1_{i+1})
\end{equation}

From (\ref{eqa2}), by definition of $R^@$, there exists a sequence $C^2_1,\ldots,C_m^2$ of elements of $S$ such that
\begin{equation}
C^2_1 = B
\end{equation}
\begin{equation}
C^2_m = C
\end{equation}
\begin{equation}\label{boon}
\mbox{for every $i \in \{1..m-1\}$,} R(C^2_i,C^2_{i+1})
\end{equation}

\noindent
Consider the sequence $D  = C^1_1,\ldots,C^1_n,\ldots,C^2_m $.
It is easy to check using (\ref{boo}) - (\ref{boon})  that $D$ satisfies conditions from the definition of $R^@$ for the pair $(A,C)$. Therefore, (\ref{eqa3}) holds and $R^@$ is transitive. 
Therefore, from (\ref{eqr1}) - (\ref{eqr3}), $R^@$ is a reflexive transitive relation containing $R$. By definition of $R^+$, we must have:

\begin{equation}\label{slr}
R^+ \subseteq R^@
\end{equation}

\noindent
We next show
\begin{equation}\label{srl}
R^@ \subseteq R^+
\end{equation}

Let $(A,B) \in R^@$. Then, by definition of $R^@$, there exists a sequence $C_1,\ldots,C_k$ of elements of $S$ such that 
 
\begin{equation}
C_1 = A
\end{equation}
\begin{equation}
C_k = B
\end{equation}
\begin{equation}\label{boon1}
\mbox{for every $i \in \{1..k-1\}$,} R(C_i,C_{i+1})
\end{equation}

We prove the following:
\begin{equation}\label{ind}
R^+(C_1,C_r) \mbox{ for every $r\in\{1,\ldots,k\}$}
\end{equation}

We prove (\ref{ind}) by induction on $r$.
\begin{itemize}
\item[\textbf{Base Case}] For $r = 1$, $R^+(C_1,C_1)$ holds because $R^+$ is a reflexive relation on $S$ by \red{clause 3 of the definition}.
\item[\textbf{Induction Hypothesis}] Suppose $R^+(C_1,C_l)$ for some $l \in\{1..k-1\}$
\item[\textbf{Inductive Step}] We need to show 
\begin{equation}
R^+(C_1,C_{l+1})
\end{equation} 
From (\ref{boon1}) we have 
\begin{equation}\label{tr1}
R^+(C_l, C_{l+1})
\end{equation}
 By inductive hypothesis, we have:
\begin{equation}\label{tr2}
R^+(C_1,C_l)
\end{equation}
Since $R^+$ is a transitive relation by \red{clause 3 of the definition}, from (\ref{tr1}) and (\ref{tr2}) we have:
\begin{equation}
R^+(C_1,C_{l+1})
\end{equation}

\end{itemize}

Therefore, (\ref{ind}) holds, and, in particular, we have
\begin{equation}
R^+(C_1, C_k)
\end{equation}
Therefore, (\ref{srl}) holds. From (\ref{srl}) and (\ref{slr}) we have
$$R^@ = R^+$$ which concludes the proof.

\end{proof}
\end{document}


%%% Local Variables:
%%% mode: latex
%%% TeX-master: t
%%% End:
