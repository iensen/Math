\documentclass[letterpaper, 11pt]{article}

\usepackage{amsmath, amsthm, latexsym, amssymb, graphicx, bold-extra, mathrsfs, frcursive}
\usepackage[pdftex]{color}
\usepackage[T1]{fontenc}

% Simplifies margin settings
\usepackage{geometry}
\geometry{margin=1in}

% Puts list item indicators in bold; makes flush with previous margin
\renewcommand\labelenumi{\bf\theenumi.}
\renewcommand\labelenumii{\bf\theenumii.}
% setlength\leftmargini{1.4em}
\setlength\leftmarginii{1.4em}

% Flexibility for headers and footers
\usepackage{fancyhdr}
\pagestyle{fancyplain}
\fancyhf{} %clear all header and footer fields
\lhead{\bf \small Probability and Statistics \hspace*{\fill} Page \thepage}
\headsep 0.2in
\thispagestyle{empty}
\renewcommand{\headrulewidth}{0pt}
\renewcommand{\footrulewidth}{0pt}

\parindent 0in
\parskip 10pt
\setlength{\headheight}{20pt}

\title{CLRS Problem  12-4b Solution}

\begin{document}

%=======================================

\begin{center}
\Large \bf CLRS 3-ed, Page 306, Problem 12-4b

\large Evgenii Balai
\end{center}

\textbf{Problem Statement}\\
Show that  $B(x) =x B(x) ^2 +1$

\bigskip

\textbf{Solution}\\
\begin{align*}
B(X) &= \displaystyle\sum_{n=0}^{\infty}b_{n}x^n\\
     &=b_0 + \displaystyle\sum_{n=1}^{\infty}b_{n}x^n\\
     &=b_0+x\displaystyle\sum_{n=1}^{\infty}\displaystyle\sum_{k=0}^{n-1}b_kb_{n-k-1}x^{n-1}\\
     &=b_0+x\displaystyle\sum_{n=0}^{\infty}\displaystyle\sum_{k=0}^{n}b_kb_{n-k}x^{n}\\ 
     &=b_0+x\displaystyle\sum_{n=0}^{\infty}b_nx_n \cdot \sum_{n=0}^{\infty}b_nx_n \mbox{~(check this by computing the coefficient for every term $x^i$)}\\
     &=b_0 +x B(X)^2\\
     &=1 + xB(X)^2\\
\end{align*}





\clearpage

%=======================================

\end{document}